\documentclass[10pt]{article}
%\documentclass[review]{siamart1116}
%\documentclass[review]{siamonline1116}

\usepackage{amsmath}
\usepackage{amsfonts}
\usepackage{amssymb}
\usepackage{fancyhdr}
\usepackage[margin=0.75in]{geometry}
\usepackage{graphicx}
\usepackage[section]{placeins}
\usepackage{nicefrac}
\usepackage{bm}
\usepackage{xcolor}
\usepackage[format=plain,indention=.5cm, font={it,small},labelfont=bf]{caption}
\usepackage{subcaption}
\usepackage{float}
\usepackage{enumerate}
\usepackage{tikz}
\usetikzlibrary{arrows.meta}
\usepackage[all]{xy}
\usepackage{url}
\usepackage{multicol}
\usepackage{cprotect}



%%%%%%%%%%%%%%%%%%%%%%%%%%%%%%%%%%%%%%%%%%%%%%%%%%%%%%%%%%%%%%%%%%%%%%%%%%%%%%%%%%%%%%%%%%%%%%%%%%%%%%%%%%%%%%%%%%
\usepackage{stackengine}
\usepackage{amsthm}
\usepackage{cleveref}


\newtheorem{theorem}{Theorem}
\newtheorem*{theorem*}{Theorem}
\newtheorem{lemma}{Lemma}
\newtheorem*{lemma*}{Lemma}
\newtheorem{conj}{Conjecture}
\newtheorem{corollary}{Corollary}
\newtheorem{clm}{Claim}
\newtheorem{rmk}{Remark}
\newtheorem{note}{NOTE}
\newtheorem{method}{Method}

\theoremstyle{definition}
\newtheorem*{def*}{Definition}
\newtheorem{definition}{Definition}
\numberwithin{theorem}{section}
\numberwithin{definition}{section}
\numberwithin{lemma}{section}
\numberwithin{corollary}{section}
\numberwithin{clm}{section}
\numberwithin{rmk}{section}

\newcommand{\low}[1]{$_{\text{#1}}$}
\newcommand\xput[2][0.5]{%
	\rule{#1\linewidth}{0pt}\makebox[0pt][c]{#2}\hfill}

\setlength{\headheight}{15pt}
\pagestyle{fancy}
\renewcommand{\headrulewidth}{0pt}
\fancyhead[L]{Brunner}
\fancyhead[C]{Co-Occurrance Network}
\fancyhead[R]{\today}
\lfoot{}
\cfoot{\thepage}
\rfoot{}

%%%%%%%%%%%%%%%%%%%%%%%%%%%%%%%%%%%%%%%%%%%%%%%%%%%%%%%%%%%%%%%%%%%%%%%%%%%%%%%%%%%%%%%%%%%%%%%%%%%%%%%%%%%%%%%%%%
%
%
%\newsiamthm{clm}{Claim}
%\newsiamremark{rmk}{Remark}
%\newsiamremark{note}{NOTE}
%\numberwithin{theorem}{section}
%
%
%%%%%%%%%%%%%%%%%%%%%%%%%%%%%%%%%%%%%%%%%%%%%%%%%%%%%%%%%%%%%%%%%%%%%%%%%%%%%%%%%%%%%%%%%%%%%%%%%%%%%%%%%%%%%%%%%%
\newenvironment{inbox}[1]
{\begin{center}
		\begin{tabular}{|p{0.9\textwidth}|}
			\hline
			{\bf #1}\\
		}
		{ 
			\\\\\hline
		\end{tabular} 
	\end{center}
}
\newenvironment{inbox2}
{\begin{center}
		\begin{tabular}{|p{0.9\textwidth}|}
			\hline \vspace{-0.5 cm}
		}
		{ 
			\\ \hline
		\end{tabular} 
	\end{center}
}




\newcommand{\nhalf}{\nicefrac{1}{2}}
\newcommand{\eps}{\epsilon_{machine}}
\newcommand{\ol}{\overline}
\renewcommand{\b}{\bm}

\definecolor{dgreen}{RGB}{49,128,23}
\definecolor{lgreen}{RGB}{77, 255, 166}
\definecolor{nicepink}{RGB}{255, 0, 102}
\definecolor{nicered}{RGB}{255, 80, 80}
\definecolor{lblue}{RGB}{102, 163, 255}
\definecolor{lgray}{RGB}{217, 217, 217}

\newcommand{\bE}{\mathbb{E}}
\newcommand{\bP}{\mathbb{P}}
\newcommand{\bR}{\mathbb{R}}
\newcommand{\bN}{\mathbb{N}}
\newcommand{\bZ}{\mathbb{Z}}
\newcommand{\bQ}{\mathbb{Q}}
\newcommand{\bC}{\mathbb{C}}
\newcommand{\cA}{\mathcal{A}}
\newcommand{\cB}{\mathcal{B}}
\newcommand{\cC}{\mathcal{C}}
\newcommand{\cD}{\mathcal{D}}
\newcommand{\cE}{\mathcal{E}}
\newcommand{\cF}{\mathcal{F}}
\newcommand{\cG}{\mathcal{G}}
\newcommand{\cH}{\mathcal{H}}
\newcommand{\cI}{\mathcal{I}}
\newcommand{\cJ}{\mathcal{J}}
\newcommand{\cK}{\mathcal{K}}
\newcommand{\cL}{\mathcal{L}}
\newcommand{\cM}{\mathcal{M}}
\newcommand{\cN}{\mathcal{N}}
\newcommand{\cO}{\mathcal{O}}
\newcommand{\cP}{\mathcal{P}}
\newcommand{\cQ}{\mathcal{Q}}
\newcommand{\cR}{\mathcal{R}}
\newcommand{\cS}{\mathcal{S}}
\newcommand{\cT}{\mathcal{T}}
\newcommand{\cU}{\mathcal{U}}
\newcommand{\cV}{\mathcal{V}}
\newcommand{\cW}{\mathcal{W}}
\newcommand{\cX}{\mathcal{X}}
\newcommand{\cY}{\mathcal{Y}}
\newcommand{\cZ}{\mathcal{Z}}

\newcommand{\inter}{\text{\normalfont int}}
\newcommand{\ka}{\kappa}
\newcommand{\fp}{\varrho}
\newcommand{\problem}[2]{ \ \\ {\bf #1} {\it #2} \ \\} 

\renewcommand{\arraystretch}{1.5}
\renewcommand{\thefootnote}{\fnsymbol{footnote}}	
\author{Jim Brunner}
\title{Navigating This Repo}

\begin{document}
\maketitle

\section{Parent Folder}
The parent folder contains all the python and bash scripts used in the project, as well as the python module of functions. The python module of function is \verb|co_occ_funs.py|. The python scripts are:
\begin{itemize}
	\item \verb|adding_data.py| - Experiment to test network building with growing data set.
	\item \verb|cluster_net.py| - Clustering for network that has been built.
	\item \verb|co_occurrence.py| - Creates Networks.
	\item \verb|examples.py| - Makes examples for a paper.
	\item \verb|falsepm.py| - Validation experiments.
	\item \verb|mc_speed.py| - Test time to build networks.
	\item \verb|network_stats.py| - Compares various statistics of a network to the null model.
	\item \verb|rand_samp.py| - Creates a ``sample" that can be analyzed.
	\item \verb|sample_analysis.py| - Analyses a sample, creates a table that can be added to a cytoscape network.
\end{itemize}

The bash scripts are
\begin{itemize}
	\item \verb|net_making.sh| - Runs \verb|co_occurrence.py|, \verb|cluster_net.py|, \verb|network_stats.py|, and \verb|falsepm.py|, for genus and species level. Creates a folder with the date and time: \verb|MM_DD_HH_MM_networks|, and subfolders for taxonomic level and network creation method. 
	\item \verb|sample_tests.sh| - runs \verb|rand_samp.py| and \verb|sample_analysis.py|
\end{itemize}

For detailed code documentation, see \verb|writeups/docs/code_docs.pdf|.

In addition, the parent folder contains the following text files:
\begin{itemize}
	\item \verb|count_by_type.txt| - Contains a count of the sample types in \verb|merged2.txt|, and the holdouts from \verb|08_14_14_57_networks|
	\item \verb|merged_assignment.txt| - Small set of GOTTCHA output data (from HMP?)
	\item \verb|merged2.txt| - Larger set of HMP data.
\end{itemize}

Also, it contains this file and the various auxiliary files used to create it.

\section{pycache}
This folder is created by python when a module is created (in this case \verb|co_occ_funs.py| and can be ignored.

\cprotect \section{\verb|08_08_17_09_networks| \& \verb|08_14_14_57_networks|}
These folders contain network files that can be put into cytoscape. The numbers are date and time in \verb|MM_DD_HH_MM| format. In each folder there two subfolders corresponding to taxonomic level. In each of these, there is again two subfolders corresponding to the network building method. The \verb|bins| folder contains networks built using the ``binning" method, and the \verb|pears| folder contains networks built using Pearson correlation. 

All network files are \verb|.tsv| text files, and can be read into a python environment using pandas \verb|read_csv('filename', sep = '\t')|. For each network, there are four files, which differ in the end of the filename:
\begin{itemize}
	\item \verb|adj.tsv| - This is the adjacency matrix, used for any analysis you want to do in python, not for cytoscape
	\item \verb|list.tsv| - This is a list of edges, and can be imported into cytoscape using \verb|Import Network From File|.
	\item \verb|node_data.tsv| - This is a node data table and can be added to the existing cytoscape network using \verb|Import Table From File|. 
	\item \verb|held.tsv| - This is a list of the data columns that were not used in creating the network.
\end{itemize}

The procedure for importing a network into cytoscape is then:
\begin{enumerate}
	\item Click ``Import Network From File" (a down arrow pointing at a cartoon of a 3 node network) and click a the network file ending in \verb|list.tsv|. Make sure there isn't an index column in the import box that comes up (if there is, click the dropdown above the column and select ``not imported"). 
	\item Click ``Import Table From File" (a down arrow pointing at a cartoon of a spreadsheet) and select the file ending in \verb|node_data.tsv|\cprotect \emph{that has the same beginning of the file name as the \verb|list.tsv| file}. In the resulting import box, at the top dropdown, change the selection next to ``Where To Import Table Data" to ``Selected Networks Only", and select the correct network. Make sure the first column (with a key above it) is a list of taxa names. If there is an index list, don't import it and put the key above the list of taxa names.
	\item If you want colors, go to ``style", find the color attribute you want (node fill, for example), select ``passthrough mapping", and select a data column containing the word ``color".
\end{enumerate}

\subsection{pears}
In the \verb|pears| folder there is a couple of extra things. These are: a \verb|stats.txt| file that contains the results of \verb|network_stats.py|, and folder called \verb|validation_plots|. This folder contains the results of \verb|falsepm.py|.

\section{Old Scripts}
This folder contains some old code that might not work anymore because of changes to \verb|co_occ_funs.py|. 
\begin{itemize}
	\item \verb|add_gender.py| - added the gender column to node data in \verb|08_08_17_09_networks| 
	\item \verb|cleanup.py| - cleaned up network files for easier import into cytoscape. No longer needed, this cleanup has been incorporated into \verb|co_occurrence.py|.
	\item \verb|color_key.py| - creates a plot of colors. Can be used to create a key for the coloring of nodes that is in the node data table.
\end{itemize}

\section{RCode}
Pavel's folder - contains metadata for data I don't have

\section{Stat Figs}
Contains figures made. The files in the folder are made from various experiments using \verb|merged_assignment.txt| data, while the folder \verb|merged2| used data from \verb|merged2.txt|. 

\section{Test Samples}
This folder contains ``samples" that can be put through \verb|sample_analysis.py| to make examples.

\section{Writeups}
This contains all the documentation. The files are picturs (\verb|.png| files) and one bibliography file (\verb|.bib| file) which is used to add references in a \LaTeX \,document. Documentation was created in \LaTeX, meaning numerous auxiliary files are created. The \verb|.tex| writes the document, and the \verb|.pdf| file is the what should be read. The other files can be ignored. To edit a document, open the \verb|.tex| file and write away. Must have \LaTeX \,installed on the computer, and I like TeXstudio as an editor.

\subsection{docs}
Code documentation. 

\subsection{notes}
All notes from the summer related directly to this project. This document contains some ideas that I haven't explored.

\subsection{paper}
The paper on the project.

\subsection{presentation}
Presentation for the project (also created in \LaTeX using beamer).


\end{document}