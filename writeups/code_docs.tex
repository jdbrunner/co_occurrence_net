\documentclass[10pt]{article}
%\documentclass[review]{siamart1116}
%\documentclass[review]{siamonline1116}

\usepackage{amsmath}
\usepackage{amsfonts}
\usepackage{amssymb}
\usepackage{fancyhdr}
\usepackage[margin=0.75in]{geometry}
\usepackage{graphicx}
\usepackage[section]{placeins}
\usepackage{nicefrac}
\usepackage{bm}
\usepackage{xcolor}
\usepackage[format=plain,indention=.5cm, font={it,small},labelfont=bf]{caption}
\usepackage{subcaption}
\usepackage{float}
\usepackage{enumerate}
\usepackage{tikz}
\usetikzlibrary{arrows.meta}
\usepackage[all]{xy}
\usepackage{url}
\usepackage{multicol}
\usepackage{cprotect}



%%%%%%%%%%%%%%%%%%%%%%%%%%%%%%%%%%%%%%%%%%%%%%%%%%%%%%%%%%%%%%%%%%%%%%%%%%%%%%%%%%%%%%%%%%%%%%%%%%%%%%%%%%%%%%%%%%
\usepackage{stackengine}
\usepackage{amsthm}
\usepackage{cleveref}


\newtheorem{theorem}{Theorem}
\newtheorem*{theorem*}{Theorem}
\newtheorem{lemma}{Lemma}
\newtheorem*{lemma*}{Lemma}
\newtheorem{conj}{Conjecture}
\newtheorem{corollary}{Corollary}
\newtheorem{clm}{Claim}
\newtheorem{rmk}{Remark}
\newtheorem{note}{NOTE}
\newtheorem{method}{Method}

\theoremstyle{definition}
\newtheorem*{def*}{Definition}
\newtheorem{definition}{Definition}
\numberwithin{theorem}{section}
\numberwithin{definition}{section}
\numberwithin{lemma}{section}
\numberwithin{corollary}{section}
\numberwithin{clm}{section}
\numberwithin{rmk}{section}

\newcommand{\low}[1]{$_{\text{#1}}$}
\newcommand\xput[2][0.5]{%
	\rule{#1\linewidth}{0pt}\makebox[0pt][c]{#2}\hfill}

\setlength{\headheight}{15pt}
\pagestyle{fancy}
\renewcommand{\headrulewidth}{0pt}
\fancyhead[L]{Brunner}
\fancyhead[C]{Co-Occurrance Network}
\fancyhead[R]{\today}
\lfoot{}
\cfoot{\thepage}
\rfoot{}

%%%%%%%%%%%%%%%%%%%%%%%%%%%%%%%%%%%%%%%%%%%%%%%%%%%%%%%%%%%%%%%%%%%%%%%%%%%%%%%%%%%%%%%%%%%%%%%%%%%%%%%%%%%%%%%%%%
%
%
%\newsiamthm{clm}{Claim}
%\newsiamremark{rmk}{Remark}
%\newsiamremark{note}{NOTE}
%\numberwithin{theorem}{section}
%
%
%%%%%%%%%%%%%%%%%%%%%%%%%%%%%%%%%%%%%%%%%%%%%%%%%%%%%%%%%%%%%%%%%%%%%%%%%%%%%%%%%%%%%%%%%%%%%%%%%%%%%%%%%%%%%%%%%%
\newenvironment{inbox}[1]
{\begin{center}
		\begin{tabular}{|p{0.9\textwidth}|}
			\hline
			{\bf #1}\\
		}
		{ 
			\\\\\hline
		\end{tabular} 
	\end{center}
}
\newenvironment{inbox2}
{\begin{center}
		\begin{tabular}{|p{0.9\textwidth}|}
			\hline \vspace{-0.5 cm}
		}
		{ 
			\\ \hline
		\end{tabular} 
	\end{center}
}




\newcommand{\nhalf}{\nicefrac{1}{2}}
\newcommand{\eps}{\epsilon_{machine}}
\newcommand{\ol}{\overline}
\renewcommand{\b}{\bm}

\definecolor{dgreen}{RGB}{49,128,23}
\definecolor{lgreen}{RGB}{77, 255, 166}
\definecolor{nicepink}{RGB}{255, 0, 102}
\definecolor{nicered}{RGB}{255, 80, 80}
\definecolor{lblue}{RGB}{102, 163, 255}
\definecolor{lgray}{RGB}{217, 217, 217}

\newcommand{\bE}{\mathbb{E}}
\newcommand{\bP}{\mathbb{P}}
\newcommand{\bR}{\mathbb{R}}
\newcommand{\bN}{\mathbb{N}}
\newcommand{\bZ}{\mathbb{Z}}
\newcommand{\bQ}{\mathbb{Q}}
\newcommand{\bC}{\mathbb{C}}
\newcommand{\cA}{\mathcal{A}}
\newcommand{\cB}{\mathcal{B}}
\newcommand{\cC}{\mathcal{C}}
\newcommand{\cD}{\mathcal{D}}
\newcommand{\cE}{\mathcal{E}}
\newcommand{\cF}{\mathcal{F}}
\newcommand{\cG}{\mathcal{G}}
\newcommand{\cH}{\mathcal{H}}
\newcommand{\cI}{\mathcal{I}}
\newcommand{\cJ}{\mathcal{J}}
\newcommand{\cK}{\mathcal{K}}
\newcommand{\cL}{\mathcal{L}}
\newcommand{\cM}{\mathcal{M}}
\newcommand{\cN}{\mathcal{N}}
\newcommand{\cO}{\mathcal{O}}
\newcommand{\cP}{\mathcal{P}}
\newcommand{\cQ}{\mathcal{Q}}
\newcommand{\cR}{\mathcal{R}}
\newcommand{\cS}{\mathcal{S}}
\newcommand{\cT}{\mathcal{T}}
\newcommand{\cU}{\mathcal{U}}
\newcommand{\cV}{\mathcal{V}}
\newcommand{\cW}{\mathcal{W}}
\newcommand{\cX}{\mathcal{X}}
\newcommand{\cY}{\mathcal{Y}}
\newcommand{\cZ}{\mathcal{Z}}

\newcommand{\inter}{\text{\normalfont int}}
\newcommand{\ka}{\kappa}
\newcommand{\fp}{\varrho}
\newcommand{\problem}[2]{ \ \\ {\bf #1} {\it #2} \ \\} 

\renewcommand{\arraystretch}{1.5}
\renewcommand{\thefootnote}{\fnsymbol{footnote}}	
\author{Jim Brunner}
\title{Co-occurrence Code Documentation}

\begin{document}
\maketitle

\tableofcontents

\cprotect \section{\verb|co_occ_funs.py|}
Module of python functions used in the rest of the project.
\cprotect \subsection{\verb|both_occ|}
Function that counts the number of times both \verb|r1| and \verb|r2| are within a range, and then 
returns the fraction of times this occurs. The range is half open: $(a,b]$.
\begin{itemize}
	\item Input: array \verb|r1| and array \verb|r2|.
	\item Optional input: lower bound \verb|lbd|, default $0$. Upper bound \verb|ubd| default $1$.
	\item Output: int count of indices $i$ such that $r1(i) \in (lbd, ubd]$ and $r2(i) \in (lbd, ubd]$.
\end{itemize}

\cprotect \subsection{\verb|both_same_bin|}
Function that counts the number of times the two occur in the same abundance range.
\begin{itemize}
	\item Input: array \verb|r1|, array \verb|r2|, float \verb|lthresh|, int \verb|numthresh|.
	\item Optional input: bool \verb|rel|, default True.
	\item Output: count of indices $i$ such that such that $r1(i) \in (lbd_j, ubd_j]$ and $r2(i) \in (lbd_j, ubd_j]$ for $j = 0,..,numthresh$.
\end{itemize}

Divides the interval $[0,1]$ into $numthresh$ intervals $(a_j,b_j]$ and calls \verb|both_occ(r1,r2,lbd = a_j, ubd = b_j)| for each. 

\cprotect \subsection{\verb|color_picker|}
Function that classifies nodes by which type of sample they have the highest abundance in. If it's a dataframe it will return the column head of the winner'

\begin{itemize}
	\item Input: array or row of dataframe \verb|r|
	\item Optional input: none
	\item Output: index or column name of argmax of \verb|r|, if the max is greater than $1.5*\sigma  + \mu$, where $\sigma$ is the sample standard deviation, and $\mu$ is the sample mean
\end{itemize}

\cprotect \subsection{\verb|matchyn|}

\begin{itemize}
	\item Input: scalars or list \verb|a|, \verb|b|
	\item Optional input: none
	\item Output: \verb|a| if $a = b$, otherwise \verb|['Intertype', grey]|, where grey is the hex value for grey
\end{itemize}

purpose is to compare two lists that look like \verb|['Class', color]|. 

\cprotect \subsection{\verb|occ_probs|}
Calculate probability of occurrence at an abundance level in a random graph, binomial distribution with parameters edge degree and sample degree/total edges, as in \cite{coocc}.

\begin{itemize}
	\item Input: dataframe of GOTTCHA output \verb|abund_array|, lower bound \verb|lthresh|, number of bins \verb|numthresh|
	\item Optional input: bool rel, default True - whether or not to normalize GOTTCHA data.
	\item Output: array \verb|occ_prob|, the probability of an abundance level in a null model that assumes abundances are binomial with parameters coming from the number of taxa appearing in the sample and number of times a taxa appears.
\end{itemize}

\cprotect \subsection{\verb|random_coocc_prob|}
Calculate a poisson-binomial...$P(X > wij)$ where $X$ is the number of times i and j co occur in random graph (X is a random variable) \cite{coocc}.
\begin{itemize}
	\item Input: array of probabilities (output of \verb|occ_probs|) \verb|occ|, \verb|wij| the abundance to test, \verb|i|, \verb|j|, the indices of the taxa pair tested
	\item Optional input: none
	\item Output: float, probability that the null model produces taxa i and taxa j in the same bin more than $wij$ times.
\end{itemize}

First gets the probability for each sample-level they are both present, then calculates a probability of the number of times this happens. Very very slow, do not use. This is because there are many ways 2 things can co-occur some number of times.

\cprotect \subsection{\verb|approx_rand_prob|}
Calculate an approximation of a poisson-binomial...P(X > wij) where X 
is the number of timesi and j co occur in random graph (X is a random variable)
The paper \cite{coocc} calls it bi-binomial because its as if you had two 
probabilities in the above.

\begin{itemize}
	\item Input: array of probabilities (output of \verb|occ_probs|) \verb|occ|, \verb|wij| the abundance to test, \verb|i|, \verb|j|, the indices of the taxa pair tested
	\item Optional input: none
	\item Output: float, approximate probability that the null model produces taxa i and taxa j in the same bin more than $wij$ times.
\end{itemize}

First gets the probability for each sample-level they are both present, then calculates an approximate probability.

\cprotect \subsection{\verb|mc_pearson|}

MC approximation for pearson coefficient where X is a random vector of binomial(n1,p1)
and Y is a random vector of binomial(n2,p2), where p1,p2 are vectors. Expected value is identity matrix.

\begin{itemize}
	\item Input: N - matrix of number of trials, P - matrix of probability of successes, W - the observed correlation matrix
	\item Optional input: \verb|num_samps|, default = 1000. Number of Monte Carlo draws
	\item Output: matrix giving the percentage of MC draws that had higher correlation than observed.
\end{itemize}

If $n$ is the number of samples $m$ the number of taxa in the network being built (number of taxa in data at given level, minus any that do not correlate with any other taxa), then $N$ and $P$ should be $m\times n$ matrices. The MC draw is an $m\times m$ matrix of ``correlations" made from the null model.

\cprotect \subsection{\verb|make_null|}

Companion to \verb|mc_pearson| for palatalization

\begin{itemize}
	\item Input: N, P, W from \verb|mc_pearson|
	\item Optional input: none
	\item Output: sample - A binary matrix of bools (simulated correlation > observed correlation)
\end{itemize}

The null model assumes that the abundance of a taxa in a sample is binomial(n,p) where $n$ is the number of times the taxa is seen in the real data, and $p$ is the proportion of non-zero entries in the real data that occur in the sample. A simulated data array of abundances of taxa in sample is created. Then, the correlation matrix is computed for this. Finally, this is compared to the observed correlation matrix.

\cprotect \subsection{\verb|mc_pearson_thr|}
MC approximation for pearson coefficient where X is a random vector of binomial(n1,p1)
and Y is a random vector of binomial(n2,p2), where p1,p2 are vectors. Expected value is identity.
\begin{itemize}
	\item Input: N - matrix of number of trials, P - matrix of probability of successes, W - the observed correlation matrix
	\item Optional input: \verb|num_samps|, default = 1000. Number of Monte Carlo draws
	\item Output: matrix giving the percentage of MC draws that had higher correlation than observed.
\end{itemize}

Same as the two above (in combination), with the only difference being that simulated data is thresholded before correlations are computed. Should compare to observed correlations of thresholded data. 

\cprotect \subsection{\verb|min_nz|}

\begin{itemize}
	\item Input: 
	\item Optional input:
	\item Output: 
\end{itemize}

\cprotect \subsection{\verb|build_network|}

\begin{itemize}
	\item Input: 
	\item Optional input:
	\item Output: 
\end{itemize}

\cprotect \subsection{\verb|make_meta|}

\begin{itemize}
	\item Input: 
	\item Optional input:
	\item Output: 
\end{itemize}

\cprotect \subsection{\verb|make_meta_from_file|}

\begin{itemize}
	\item Input: 
	\item Optional input:
	\item Output: 
\end{itemize}

\cprotect \subsection{\verb|mc_network_stats|}

\begin{itemize}
	\item Input: 
	\item Optional input:
	\item Output: 
\end{itemize}

\cprotect \subsection{\verb|sim_pears|}

\begin{itemize}
	\item Input: 
	\item Optional input:
	\item Output: 
\end{itemize}

\cprotect \subsection{\verb|sim_pears_thr|}

\begin{itemize}
	\item Input: 
	\item Optional input:
	\item Output: 
\end{itemize}

\cprotect \subsection{\verb|sim_bins|}

\begin{itemize}
	\item Input: 
	\item Optional input:
	\item Output: 
\end{itemize}

\cprotect \subsection{\verb|edge_prob|}

\begin{itemize}
	\item Input: 
	\item Optional input:
	\item Output: 
\end{itemize}

\cprotect \subsection{\verb|nodes_in_sub|}

\begin{itemize}
	\item Input: 
	\item Optional input:
	\item Output: 
\end{itemize}

\cprotect \subsection{\verb|random_sub_graph|}

\begin{itemize}
	\item Input: 
	\item Optional input:
	\item Output: 
\end{itemize}

\cprotect \subsection{\verb|exp_cut_edges|}

\begin{itemize}
	\item Input: 
	\item Optional input:
	\item Output: 
\end{itemize}

\cprotect \subsection{\verb|cut_cond|}

\begin{itemize}
	\item Input: 
	\item Optional input:
	\item Output: 
\end{itemize}

\cprotect \subsection{\verb|com_clust|}

\begin{itemize}
	\item Input: 
	\item Optional input:
	\item Output: 
\end{itemize}

\cprotect \subsection{\verb|spectral_cluster|}

\begin{itemize}
	\item Input: 
	\item Optional input:
	\item Output: 
\end{itemize}

\cprotect \subsection{\verb|clust_judge|}

\begin{itemize}
	\item Input: 
	\item Optional input:
	\item Output: 
\end{itemize}

\cprotect \subsection{\verb|color_picker2|}

\begin{itemize}
	\item Input: 
	\item Optional input:
	\item Output: 
\end{itemize}

\cprotect \subsection{\verb|est_prob|}

\begin{itemize}
	\item Input: 
	\item Optional input:
	\item Output: 
\end{itemize}

\cprotect \subsection{\verb|find_cliques|}

\begin{itemize}
	\item Input: 
	\item Optional input:
	\item Output: 
\end{itemize}

\cprotect \subsection{\verb|psi_over_psi|}

\begin{itemize}
	\item Input: 
	\item Optional input:
	\item Output: 
\end{itemize}

\cprotect \subsection{\verb|diff_cliques|}

\begin{itemize}
	\item Input: 
	\item Optional input:
	\item Output: 
\end{itemize}

\cprotect \subsection{\verb|diffusion_ivp|}

\begin{itemize}
	\item Input: 
	\item Optional input:
	\item Output: 
\end{itemize}

\cprotect \subsection{\verb|diffusion_bvp|}

\begin{itemize}
	\item Input: 
	\item Optional input:
	\item Output: 
\end{itemize}

\cprotect \subsection{\verb|diffusion_forced|}

\begin{itemize}
	\item Input: 
	\item Optional input:
	\item Output: 
\end{itemize}

\cprotect \subsection{\verb|ivp_score|}

\begin{itemize}
	\item Input: 
	\item Optional input:
	\item Output: 
\end{itemize}

\cprotect \subsection{\verb|make_sample|}

\begin{itemize}
	\item Input: 
	\item Optional input:
	\item Output: 
\end{itemize}

\cprotect \subsection{\verb|get_sample|}

\begin{itemize}
	\item Input: 
	\item Optional input:
	\item Output: 
\end{itemize}

\cprotect \subsection{\verb|flat_two_deep|}

\begin{itemize}
	\item Input: 
	\item Optional input:
	\item Output: 
\end{itemize}

\cprotect \subsection{\verb|flat_one_deep|}

\begin{itemize}
	\item Input: 
	\item Optional input:
	\item Output: 
\end{itemize}

\cprotect \section{\verb|co_occurrence.py|}

\cprotect \section{\verb|cluster_net.py|}

\cprotect \section{\verb|network_stats.py|}

\cprotect \section{\verb|falsepm.py|}

\cprotect \section{\verb|sample_analysis.py|}

\cprotect \section{\verb|examples.py|}

\cprotect \section{\verb|adding_data.py|}

\cprotect \section{\verb|mc_speed.py|}

\cprotect \section{\verb|add_gender.py|}

\cprotect \section{\verb|cleanup.py|}

\cprotect \section{\verb|color_key.py|}

\cprotect \section{\verb|funct_tests.py|}

\cprotect \section{\verb|rand_samp.py|}

\cprotect \section{\verb|net_making.sh|}

\cprotect \section{\verb|sample_test.sh|}

\bibliographystyle{plain}
\bibliography{../../../summer17}
\end{document}